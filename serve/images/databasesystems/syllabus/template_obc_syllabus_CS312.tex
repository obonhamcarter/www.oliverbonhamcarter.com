% !TEX TS-program = pdflatex
% !TEX encoding = UTF-8 Unicode

% This is a simple template for a LaTeX document using the "article" class.
% See "book", "report", "letter" for other types of document.

\documentclass[11pt]{article} % use larger type; default would be 10pt

\usepackage[utf8]{inputenc} % set input encoding (not needed with XeLaTeX)
\usepackage{url}
\usepackage{color}
%%% Examples of Article customizations
% These packages are optional, depending whether you want the features they provide.
% See the LaTeX Companion or other references for full information.

%%% PAGE DIMENSIONS
\usepackage{geometry} % to change the page dimensions
\geometry{a4paper} % or letterpaper (US) or a5paper or....
% \geometry{margin=2in} % for example, change the margins to 2 inches all round
% \geometry{landscape} % set up the page for landscape
%   read geometry.pdf for detailed page layout information

\usepackage{graphicx} % support the \includegraphics command and options

% \usepackage[parfill]{parskip} % Activate to begin paragraphs with an empty line rather than an indent

%%% PACKAGES
\usepackage{booktabs} % for much better looking tables
\usepackage{array} % for better arrays (eg matrices) in maths
\usepackage{paralist} % very flexible & customisable lists (eg. enumerate/itemize, etc.)
\usepackage{verbatim} % adds environment for commenting out blocks of text & for better verbatim
\usepackage{subfig} % make it possible to include more than one captioned figure/table in a single float
% These packages are all incorporated in the memoir class to one degree or another...
\usepackage{url}

%%% HEADERS & FOOTERS
\usepackage{fancyhdr} % This should be set AFTER setting up the page geometry
\pagestyle{fancy} % options: empty , plain , fancy
\renewcommand{\headrulewidth}{0pt} % customise the layout...
\lhead{}\chead{}\rhead{}
\lfoot{}\cfoot{\thepage}\rfoot{}

\long\def\omitit #1{}

%%% SECTION TITLE APPEARANCE
\usepackage{sectsty}
\allsectionsfont{\sffamily\mdseries\upshape} % (See the fntguide.pdf for font help)
% (This matches ConTeXt defaults)

%%% ToC (table of contents) APPEARANCE
\usepackage[nottoc,notlof,notlot]{tocbibind} % Put the bibliography in the ToC
\usepackage[titles,subfigure]{tocloft} % Alter the style of the Table of Contents
\renewcommand{\cftsecfont}{\rmfamily\mdseries\upshape}
\renewcommand{\cftsecpagefont}{\rmfamily\mdseries\upshape} % No bold!

%%% END Article customizations

%%% The "real" document content comes below...


\title{\textbf{CMPSC 301 -- Data Analytics\\Syllabus}}
\author{Fall 2021}
\date{}
%\date{\color{red}Syllabus updated: \today \color{black}} % Activate to display a given date or no date (if empty),
         % otherwise the current date is printed 

\tolerance=1
\emergencystretch=\maxdimen
\hyphenpenalty=10000
\hbadness=10000

\begin{document}
\maketitle

\subsection*{\textbf{Course Instructor}}
Dr. Oliver BONHAM-CARTER (said and written as ``Bonham-Carter,'' not “Carter'')\\
\noindent Email: \url{obonhamcarter@allegheny.edu} \\
\noindent Web Site: \url{http://www.cs.allegheny.edu/sites/obonhamcarter/} \\
\noindent Class and lab meeting place: Alden 101\\
\noindent Final deliverable due: 9am, 18$^{th}$ May 2022\\ %https://sites.allegheny.edu/registrar/spring-2022-final-exam-schedule/
\noindent Final Exam Code: D\\  
\noindent Syllabus updated on: \today\\

\noindent Discord channel: 
\begin{itemize} 
	\item \url{https://discord.gg/vsntc3R7}\\ \color{red}(Note: this link will expire in 7 days from 31$^{st}$ August 2021!)\color{black}\\
\end{itemize}
%	\item \url{https://discord.gg/xrRagUbe}\\ \color{red}(Note: this link will expire in 7 days from 23$^{rd}$ August 2021!)\color{black}\\

\noindent Google Calendar:
\begin{itemize}
\item {\footnotesize \url{https://calendar.google.com/calendar/u/0?cid=Y19hc2sxdW00MnNmaTdvcTk4YmwxOHE0M2xrNEBncm91cC5jYWxlbmRhci5nb29nbGUuY29t}}\\\color{red}(Note: This link is all on one line!)\color{black}\\
\end{itemize}


\subsection*{The {\tt ClassDocs/} Class Archive}

\begin{itemize}
	\item \textbf{Repository on GitHub}: 
		\begin{itemize}
			\item \footnotesize \url{https://github.com/Allegheny-ComputerScience-312-S2022/classDocs}
		\end{itemize}

	\item \textbf{HTTPS}: 
		\begin{itemize}
			\item {\tt \footnotesize git clone https://github.com/Allegheny-ComputerScience-312-S2022/classDocs.git}
		\end{itemize}

	\item \textbf{SSH}: 
		\begin{itemize}
			\item {\tt \footnotesize git clone git@github.com:Allegheny-ComputerScience-312-S2022/classDocs.git}
		\end{itemize}
	\end{itemize}



% {\color{red}Syllabus updated on: 23 August 2020}\\
%https://sites.allegheny.edu/registrar/fall-2020-final-exam-schedule/

%{\color{red} NOTE: This syllabus contains updates to respond to the college's closing due to the COVID-19 spread. These changes are in red. As always, please mail the instructor with any questions or concerns. }

\subsection*{Instructor's Office Hours}
\begin{itemize}
	\item Office hours will take place online using Zoom or in-person in my office in Alden 104. 

	\item To schedule a meeting with me during my office hours, please visit my web site and click the ``Schedule'' link in the top right-hand corner. Here, you can browse my office hours slots to schedule an appointment. If using Zoom, you will find a link with the meeting invitation. At the allotted time, I will be awaiting your meeting. If the given office hour meeting times are not convenient for your schedule, please let me know and I would be happy to work with you to find and other time which would be suitable for your schedule.
	
\end{itemize}

\color{red}

\begin{itemize}
  \itemsep 0em
  \item By Zoom or in-person in Alden 104 (please let me know which it is!)
  \item Tuesday: 2:45 pm  -- 4:45 pm EST (15 minute time slots)
  \item Wednesdays and Thursdays : 2:00pm  -- 4:00 pm EST (15 minute time slots)
  \item By appointment, if these times do not work for you.
\end{itemize}

\color{black}





\subsection*{Course Meeting Schedule}

\begin{table}
	\begin{center}
		\caption{Meetings times: lecture and lab}
		\label{tab:meeting}
			\begin{tabular}{|c|c|c|}
			\hline
			&Lecture & Lab \\
			\hline
			Meeting Time (EST) &T/Th 10:20 AM - 11:50 AM & M 2:50 PM - 4:40 PM\\
			Duration &2/21/2022 - 5/20/2022 & 2/21/2022 - 5/20/2022\\
			Where to meet &Main Campus, Alden Hall 101 & Main Campus, Alden Hall 101\\
			\hline
		\end{tabular}
	\end{center}
\end{table}


%We will be using Zoom for all class meetings. The Zoom link for class meetings is shown below.
%\begin{center}
%\footnotesize
%\url{https://allegheny.zoom.us/j/91560133944?pwd=K3FkWi8wZ2xVcEtWQWVTTGhTajdEZz09}
%\end{center}




\subsection*{\textbf{Academic Bulletin Description}}

\begin{quote}

Course Description
A study of the application and evaluation of database management systems. Participating in hands-on activities that often require teamwork, students design, implement, and deploy database systems that store interdisciplinary data sets. In addition to learning how to develop and assess interfaces for databases, students study the efficiency and effectiveness of alternative data management systems. During a weekly laboratory session students use state-of-the-art technology to complete projects, reporting on their results through both written documents and oral presentations. Students are invited to use their own departmentally approved laptop in this course; a limited number of laptops are available for use during class and lab sessions. Prerequisite: CMPSC*101. Distribution Requirements: QR (Quantitative Reasoning), SP (Scientific Process and Knowledge).

\end{quote}



\subsection*{Distribution Requirements}
The following definitions were taken from the \emph{Distribution Requirements: Learning Outcomes} website, \url{https://sites.allegheny.edu/registrar/academic-policies/graduation-requirements/distribution-requirement/distribution-requirements-learning-outcomes/}.

\begin{itemize}
	\item Quantitative Reasoning (QR). Quantitative Reasoning is the ability to understand, investigate, communicate, and contextualize numerical, symbolic, and graphical information towards the exploration of natural, physical, behavioral, or social phenomena.

	\begin{itemize}
		\item Learning Outcome: Students who successfully complete this requirement will demonstrate an understanding of how to interpret numeric data and/or their graphical or symbolic representations.
	\end{itemize}


	\item Scientific Process a Knowledge (SP). Courses involving Scientific Process and Knowledge aim to convey an understanding of what is known or can be known about the natural world; apply scientific reasoning towards the analysis and synthesis of scientific information; and create scientifically literate citizens who can engage productively in problem solving.
	
	\begin{itemize}
		\item Learning Outcome: Students who successfully complete this requirement will demonstrate an understanding of the nature, approaches, and domain of scientific inquiry.
	\end{itemize}
\end{itemize}


This course meets the course distribution requirements of QR (Quantitative Reasoning) and SP (Scientific Process and Knowledge) for its use of applying concepts of computer programming to the design and creation databases which are tested on public data from real-world applications. In addition, the class aims to introduce an component of ethical reasoning in the design, maintenance and application of database systems for potentially sensitive data.

\subsection*{Course Objectives}

The essence of the discipline of computer science is algorithms; this course will introduce students to the principles of data management using algorithms.  We will investigate some of the key techniques that scientists use to manage data. Areas of discussion include, but are not limited to, relational databases and query languages, object-oriented data storage, encoding data in the eXtensible Markup Language (XML), low-level data storage, transactions and concurrency control, data warehousing and mining, and the implementation and testing of database applications.  \\

\noindent The course will introduce students to the theory and practice of data management while covering both the well-established and the cutting-edge areas of the discipline.  The course also invites students to assess the correctness of their implementations and conduct both analytical and empirical evaluations of the performance of data
management techniques.  Moreover, the course will ask students to implement small- and medium-scale data management systems and to install and use a wide variety of support tools. In addition to improving their teamwork skills, students will enhance their ability to write and speak about software in a clear and concise fashion. Ethical discussions are also introduced into the course to introduce students to the concepts of responsible computing. 


\subsection*{Performance Objectives}

At the completion of this class, a student must be comfortable with fundamental data management topics and be aware of current research in the area.  When given a new data management problem, students should be able to select proper data management tools and implement a complete application that uses them to solve the stated problem.  Students also must develop a toolkit of data management concepts that they can use in the context of the solutions to real-world problems. Finally, students must develop and apply a strong knowledge of analytical and empirical techniques that they can use to characterize and predict the performance of data management systems.\\

\noindent Students should also be able to handle many of the important, yet accidental, aspects of implementing programs with modern programming languages and data management systems.  In addition to being comfortable with program editors, compilers, debuggers, testing tools, virtual machines, database management systems, and query languages, students will be working with some Python programming where code will be provided to be modified.



%%%%%%%%%%%%
%%%%%%%%%%%%
%%%%%%%%%%%%


\subsection*{An Ethical Interest}

Throughout the semester students will be challenged with serious analytical questions connecting the use of data, its management and storage, as well as its exploitation. The object of these discussions and challenges are to invite students to think about the responsible use of data, databases, in addition to the impacts (positive and negative) of these uses on culture, community, and society. We note here that there is often no clear indication of a ``correct'' decision concerning the application of data and its storage. Here, we will work to discuss potential outcomes from various decisions and then work together to determine which is more likely to be desirable. This work cannot give you the correct decision to make, however it can help to enable your critical thinking skills which will provide you with some understanding of how to navigate to worthy decisions.\\


\subsection*{\textbf{Textbooks}}

The material for this course will be taken from the book listed below and from the additional readings that will be provided for you. It is highly recommended that you obtain a copy of the book for your study in this course.


\begin{itemize}

\item Wickham, Hadley, and Garrett Grolemund. \emph{R for Data Science: Import, Tidy, Transform, Visualize, and Model Data.}, O'Reilly Media, Inc., 2016.
	\begin{itemize}
		\item Link to book's website: \url{http://r4ds.had.co.nz/}
	\end{itemize}

\item Silge, Julia, and David Robinson. \emph{Text mining with R: A tidy approach.} O'Reilly Media, Inc.", 2017.
\begin{itemize}
	\item Link to the books website: \url{https://www.tidytextmining.com/}
\end{itemize}

\item Along with reading the required books, you will be asked to study many additional articles from a wide variety of conference proceedings, journals, and the popular press.

\end{itemize}


\subsection*{\textbf{Students who want to improve their technical writing skills may consult the following books.}}

\begin{itemize}

	\item Crapsi, Linda. \emph{Bugs in Writing: A Guide to Debugging Your Prose.} Technical Communication 42.4 (1995): 665-667.,  ISBN-10: 020137921X, ISBN-13: 978-0201379211, 704 pages, 1998.

	\item Zobel, Justin. \emph{Writing for computer science.} Vol. 8. New York NY: Springer, 2004.,  ISBN-10: 1852338024, ISBN-13:978-1852338022, 270 pages, 2004.

\end{itemize}



\subsection*{\textbf{Class Policies}}

\subsubsection*{\textbf{Grading}}

The grade that a student receives in this class will be based on the following categories. All percentages are approximate and, if the need to do so presents itself, it is possible for the assigned percentages to change during the academic semester. Students will receive their reported grades approximately once a week through an individual GitHub grading repository.

\color{red}
\begin{center}
  \begin{tabular}{l|l}
\hline

   Class Participation & 10\% \\  
    First Examination & 10\% \\
    Second Examination & 10\% \\
    Laboratory  Assignments & 40\% \\
    Final Project & 30\% \\

\hline
  \end{tabular}
\end{center}
\color{black}
\noindent
These grading categories have the definitions which are defined below.
%Note: final exam is code A, and thus the final project is due 30$^{th}$ of April at 9:00am.\\
%\newpage


\noindent
\subsection*{\textbf{Definitions of Grading Categories}}
\vspace*{-.05in}

\begin{itemize}

  \item {\em Class Activities}: All students are required to actively participate during all of the class sessions. Your participation will mainly take a form of completing class activities that are submitted via an appropriate GitHub repository. You may also be asked to answer questions about the required reading assignments,  give presentations, and lead a discussion session in class.% and in the course's Slack channels. 



%  \item {\em First and Second Quizzes}: The quizzes are designed to permit the student to know whether she or he is ready for the exam. Although the exams will contain new material, the quizzes will contain some of the concepts which the student may expect to see on the exam. Poor scores on quizzes will alert the student to approach the subject material with more focus.
	
  \item {\em First and Second Quizzes}: The first and second quizzes will cover all of the material in their associated module(s). While the second quiz is not cumulative, it will assume that a student has a basic understanding of the material that was the focus of the first quiz. Unless prior arrangements are made with the course instructor, all students will be expected to take these quizzes on the scheduled date and complete them in the stated period of time.


%  \item {\em Final Examination}: The final examination is a three-hour cumulative test. By enrolling in this course, students agree that, unless there are extenuating circumstances, they will take the final examination at the time stated on the first page of the syllabus.

  \item {\em Laboratory Assignments}: These assignments invite students to explore the concepts, tools, and techniques associated with the analysis of data.  All of the laboratory assignments require the use of the provided tools to study, design, implement, and evaluate systems that solve data analytics problems.  In addition to demonstration of the technical skills through the utilized or developed software for data analysis, some of the laboratory assignments in this course may also expect students to read a related article and to lead a discussion or to give  a short presentation related to the assigned article.
  
    %%% Homework assignments will normally ask students to prepare short written documents reflecting on
    % facets of the software development life cycles.

  \item {\em Final Project}: This project will present you with the description of a problem and ask you to implement a full-featured solution using a wide variety of data analytics techniques. The final project in this class will require you to apply all of the knowledge and skills that you have accumulated during the course of the semester to solve a problem and, whenever possible, make your solution publicly available as a free and open-source tool. The project will invite you to draw upon both your problem solving skills and data analytics techniques. %The final project will be completed in groups assigned by the course instructor.

\end{itemize}


\subsubsection*{\textbf{Assignment Submission}}

We will be using GitHub Classroom to collect all assignments. It is expected that you are able to effectively use {\tt git} to submit your work. If you require help, please see your peers or your instructor.\\

All assignments will have a stated due date. \color{red} The electronic version of the  assignments are to be turned in at the beginning of the class session on that due date. Submissions after the beginning of class are counted as being late.   \color{black} 


\subsubsection*{\textbf{Extensions}}
Unless special arrangements are made with the course instructor, no assignments will be accepted after the late deadline. If you are requesting extensions for an assignment, then you are to email me with your request and also provide a \emph{valid reason} for your extension. This request must come before the due date of the assignment and not on the due date. Requests will not be granted where the reason appears to be insignificant. Extensions are 24 hours of extra time (after the original due date) and are given out at my discretion. The decision to provide you with an extension (or not) will be weighed in light of fairness to your peers who are still able to complete their assignments, regardless of their own busy schedules. 

The submission of homework comprises the Honor Code pledge of the student(s) completing the work. For any assignment completed in a group, students must also turn in a one-page reflection that describes each group member's contribution to the submitted deliverables.  

% All assignments will have a stated due date. The printed version of the assignment is to be turned in at the beginning
% of the class on that due date; the printed materials must be dated and signed with the Honor Code pledge of all the
% student(s) in a group.  When the printed version is submitted, the electronic version of the assignment also must be
% made available to the course instructor in a version control repository. Late assignments will be accepted for up to
% one week past the assigned due date with a 15\% penalty. All late assignments must be submitted at the beginning of
% the session that is scheduled one week after the due date. Unless special arrangements are made with the course
% instructor, no assignments will be accepted after the late deadline. In addition to submitting the required
% deliverables for any assignment completed in a group, students must turn in a one-page document that describes each
% group member's contribution to the submitted deliverables.  



\subsubsection*{\textbf{Bring your own computer to class}}

During the semester, you will be told which software to install on your machine to be prepared for class. Some of the prominent software that we may be using include;

\begin{itemize}
\item Git and GiHub (a software development software system): \url{https://github.com/}
\item Atom (an editor): \url{https://atom.io/}
\item Docker (a software container system): \url{https://www.docker.com/}
	\begin{itemize}
		\item Basic tutorial from Docker: \url{https://www.docker.com/101-tutorial}
		\item Play with Docker: \url{https://labs.play-with-docker.com/}
		\item Please note: machines running Windows ``Home" are not able to use Docker. Please verify that your machine is able to run the software by visiting the department's Approved Laptops page \url{https://www.cs.allegheny.edu/resources/laptops/}.
	\end{itemize}
	\item R (a programming language): \url{https://www.r-project.org/}
	\begin{itemize}
		\item Please see \url{https://www.r-project.org/about.html} for more information.
	\end{itemize}
	\item RStudio (An integrated development environment for R): \url{https://rstudio.com/}
	\item Information about using Docker to run RStudio: \url{https://hub.docker.com/r/rocker/rstudio}
\end{itemize}


%\subsubsection*{\textbf{Use of Laboratory Facilities}}
%Throughout the semester, we will experiment with many different tools that data managers use during the phases of the data management process.  The course instructor and the department's systems administrator have invested a considerable amount of time to ensure that our laboratories support the completion of both the laboratory assignments and the final project. To this end, students are required to complete all assignments and the final project while using the department's laboratory facilities. The course instructor and the systems administrator will only be able to devote a limited amount of time to the configuration of a student's personal computer.



\subsubsection*{\textbf{Class Preparation}}

% The study of the computer science discipline is very challenging.  Students in this class will be challenged to learn
% the principles and practice of software development.  During the coming semester even the most diligent student will
% experience times of frustration when they are attempting to understand a challenging concept or complete a difficult
% laboratory assignment.  In many situations some of the material that we examine will initially be confusing : do not
% despair!  Press on and persevere!
% 

\noindent In order to minimize confusion and maximize learning, students must invest time to prepare for class discussions and lectures.  During the class periods, the course instructor will often pose demanding questions that could require group discussion, the creation of a program, a vote on a thought-provoking issue, or a group presentation.  Only students who have prepared for class by reading the assigned material and reviewing the current assignments will be able to effectively participate in these discussions.  More importantly, only prepared students will be able to acquire the knowledge and skills that are needed to be successful in both this course and the field of data analytics.  In order to help students remain organized and effectively prepare for classes, the course instructor will maintain a class schedule. During the class sessions students will also be required to download, write, use, and modify programs, and data sets that are made available through the course GitHub repository.


\subsubsection*{\textbf{Email}}

Using your Allegheny College email address, I will sometimes send out class announcements about matters such as assignment clarifications or changes in the schedule. It is your responsibility to check your email at least once a day and to ensure that you can reliably send and receive emails. This class policy is based on the following statement in {\em The Compass}, the college's student handbook.

\vspace*{-.1in}
\begin{quote}
  ``The use of email is a primary method of communication on campus. \ldots  All students are provided with a campus email account and address while enrolled at Allegheny and are expected to check the account on a regular
  basis.'' 
\end{quote}
\vspace*{-.1in}

\subsubsection*{\textbf{Disability Services}}

The Americans with Disabilities Act (ADA) is a federal anti-discrimination statute that provides comprehensive civil rights protection for persons with disabilities. Among other things, this legislation requires all students with disabilities be guaranteed a learning environment that provides for reasonable accommodation of their disabilities. Students with disabilities who believe they may need accommodations in this class are encouraged to contact Disability Services at (814) 332-2898.  Disability Services is part of the Learning Commons and is located in Pelletier Library. Please do this as soon as possible to ensure that approved accommodations are implemented in a timely fashion.

\subsubsection*{\textbf{Honor Code}}

The Academic Honor Program that governs the entire academic program at Allegheny College is described in the Allegheny Course Catalog and in \emph{The Compass: Student Handbook}.  The Honor Program applies to all work that is submitted for academic credit or to meet non-credit requirements for graduation at Allegheny College.  This includes all work assigned for this class (e.g., examinations, laboratory assignments, and the final project).  All students who have enrolled in the College will work under the Honor Program. Each student who has matriculated at the College has acknowledged the following pledge:

\vspace*{-.1in}
\begin{quote}
\emph{I hereby recognize and pledge to fulfill my responsibilities, as defined in the Honor Code, and to maintain the integrity of both myself and the College community as a whole.}
\end{quote}
\vspace*{-.1in}

\noindent Additionally, we expect that you will adhere to the following Department Policy:

\begin{center} \textbf{ Department of Computer Science Honor Code Policy } \end{center}
\vspace*{-.1in}
It is recognized that an important part of the learning process in any course, and particularly in computer science, derives from thoughtful discussions with teachers, student assistants, and fellow students. Such dialogue is encouraged. However, it is necessary to distinguish carefully between the student who discusses the principles underlying a problem with others, and the student who produces assignments that are identical to, or merely variations on, someone else's work. It will therefore be understood that all assignments submitted to faculty of the Department of Computer Science are to be the original work of the student submitting the assignment, and should be signed in accordance with the provisions of the Honor Code. Appropriate action will be taken when assignments give evidence that they were derived from the work of others.

%%%%
omitit{
\section*{Attendance}
% ref: https://docs.google.com/document/d/1necRH1Y9JIELUUNTYyDbvkbpH9h-n1bhhFV14A9og5w/edit


\begin{itemize}

\item \textbf{Remote Attendance}
If you are participating entirely remotely this semester and relying on technology to attend class meetings, occasional technology problems that disrupt your participation will not harm your participation grade, but as with illnesses and family emergencies, chronic absences for this reason will require a more extensive discussion with me and may impact your grade.


\item \textbf{Video and Microphones}
%In addition, faculty may want to consider a statement on \emph{netiquette} (use of microphone, video, protocols for how to participate in video conference meetings).
Please turn off your microphone when not speaking during any meeting where you are using your computer. The microphone may allow for background sound to contribute to noise during the meeting. It is strongly encouraged that you use your video to show yourself during meeting. Enabling your video will allow the instructor to see hands to indicate questions. Showing video also helps to stimulate group discussions.
\end{itemize}

}% end of omitit{}



\subsection*{College Messages}
\begin{itemize}

\item \textbf{Statement of Community}
Allegheny students and employees are committed to creating an inclusive, respectful and safe residential learning community that will actively confront and challenge racism, sexism, heterosexism, religious bigotry, and other forms of harassment and discrimination. We encourage individual growth by promoting a free exchange of ideas in a setting that values diversity, trust and equality. So that the right of all to participate in a shared learning experience is upheld, Allegheny affirms its commitment to the principles of freedom of speech and inquiry, while at the same time fostering responsibility and accountability in the exercise of these freedoms.


%\item \textbf{Academic Integrity}
%Allegheny College operates under an Honor Code, to which all students are subject. See The Compass: Student Handbook. You should educate yourself appropriately as to how this applies to you. Plagiarism and other forms of intellectual dishonesty will not be tolerated.


%\item \textbf{Disability Services}
%Students with disabilities who believe they may need accommodations in this class are encouraged to contact the Office of Disability Services at (814) 332-2898.  Disability Services is located in Pelletier Library.  Please do this as soon as possible to ensure that such accommodations are implemented in a timely fashion.


\item \textbf{Learning Commons}
If you are not already, you should become familiar with the Learning Commons, located in Pelletier Library (\url{http://sites.allegheny.edu/learningcommons/}). Among other things, the staff at the Learning Commons can assist you with study and time management skills, writing, and critical reading. You should know that if you are having trouble in this class, or if I think you can specifically benefit from their services, I will refer you to the Learning Commons. Experienced peer writing and speech consultants in the Learning Commons help writers and speakers to determine strategies for effective communication and to make academically responsible choices at any stage in the writing or speaking process and on assignments in any discipline. Both appointments and drop-in sessions are available. To view the hours of operation, and to make an appointment, visit the Learning Commons website.

\item \textbf{Religious Accommodations}
If you need to miss class or reschedule a final examination due to a religious observance, please speak to the professor well in advance to make arrangements. See \url{http://sites.allegheny.edu/religiouslife/religious-holy-days/}.
\end{itemize}



\end{document}
