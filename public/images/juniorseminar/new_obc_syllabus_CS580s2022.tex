% !TEX TS-program = pdflatex
% !TEX encoding = UTF-8 Unicode

% This is a simple template for a LaTeX document using the "article" class.
% See "book", "report", "letter" for other types of document.

\documentclass[11pt]{article} % use larger type; default would be 10pt

\usepackage[utf8]{inputenc} % set input encoding (not needed with XeLaTeX)
\usepackage{url}
\usepackage{color}
%%% Examples of Article customizations
% These packages are optional, depending whether you want the features they provide.
% See the LaTeX Companion or other references for full information.

%%% PAGE DIMENSIONS
\usepackage{geometry} % to change the page dimensions
\geometry{a4paper} % or letterpaper (US) or a5paper or....
% \geometry{margin=2in} % for example, change the margins to 2 inches all round
% \geometry{landscape} % set up the page for landscape
%   read geometry.pdf for detailed page layout information

\usepackage{graphicx} % support the \includegraphics command and options

% \usepackage[parfill]{parskip} % Activate to begin paragraphs with an empty line rather than an indent

%%% PACKAGES
\usepackage{booktabs} % for much better looking tables
\usepackage{array} % for better arrays (eg matrices) in maths
\usepackage{paralist} % very flexible & customisable lists (eg. enumerate/itemize, etc.)
\usepackage{verbatim} % adds environment for commenting out blocks of text & for better verbatim
\usepackage{subfig} % make it possible to include more than one captioned figure/table in a single float
% These packages are all incorporated in the memoir class to one degree or another...
\usepackage{url}

%%% HEADERS & FOOTERS
\usepackage{fancyhdr} % This should be set AFTER setting up the page geometry
\pagestyle{fancy} % options: empty , plain , fancy
\renewcommand{\headrulewidth}{0pt} % customise the layout...
\lhead{}\chead{}\rhead{}
\lfoot{}\cfoot{\thepage}\rfoot{}

\long\def\omitit #1{}

%%% SECTION TITLE APPEARANCE
\usepackage{sectsty}
\allsectionsfont{\sffamily\mdseries\upshape} % (See the fntguide.pdf for font help)
% (This matches ConTeXt defaults)

%%% ToC (table of contents) APPEARANCE
\usepackage[nottoc,notlof,notlot]{tocbibind} % Put the bibliography in the ToC
\usepackage[titles,subfigure]{tocloft} % Alter the style of the Table of Contents
\renewcommand{\cftsecfont}{\rmfamily\mdseries\upshape}
\renewcommand{\cftsecpagefont}{\rmfamily\mdseries\upshape} % No bold!

%%% END Article customizations

%%% The "real" document content comes below...


\title{\textbf{CMPSC 580\\ Junior Seminar\\Syllabus}}
\author{Spring 2022}
\date{} % Activate to display a given date or no date (if empty),
         % otherwise the current date is printed 

\tolerance=1
\emergencystretch=\maxdimen
\hyphenpenalty=10000
\hbadness=10000

\begin{document}
\maketitle

\subsection*{\textbf{Course Instructor}}
Dr. Oliver BONHAM-CARTER (said and written as ``Bonham-Carter,'' not “Carter'')\\
\noindent Email: \url{obonhamcarter@allegheny.edu} \\
\noindent Web Site: \url{http://www.cs.allegheny.edu/sites/obonhamcarter/} \\
\noindent Class and lab meeting place: Alden 101\\
\noindent Exam Code: E\\
\noindent Final deliverable due: 18$^{th}$ May 2021 at 2:00pm\\
\noindent Distribution Requirements: \emph{none}\\
\noindent Syllabus updated on: \today\\

%\noindent Slack Team: \url{cs580Spring2017.slack.com}

\subsection*{\textbf{Instructor's Office Hours}}

\begin{itemize}
  \itemsep 0em
  \item Monday and Wednesday: 11:00am -- 12:00pm (10 minute time slots)
  \item Tuesday and Thursday: 3:00pm -- 5:00pm (10 minute time slots)
  \item By appointment
\end{itemize}


\noindent
To schedule a meeting with me during my office hours, please visit my Web site and click the ``Schedule'' link in the top right-hand corner. Now, you can view my calendar or by clicking ``schedule an appointment'' link browse my office hours and schedule an appointment by clicking the correct link to reserve an open time slot. 


\subsection*{\textbf{Technical Leaders}}
	\begin{itemize}
		\item   \url{https://www.cs.allegheny.edu/teaching/technicalleaders/}
	\end{itemize}


\subsection*{\textbf{Course Meeting Schedule}}


\textbf{Lecture, Discussion, Presentations, and Group Work}:\\
\noindent
Duration: 21 Feb 2021 - 20 May 2021\\
Tuesdays and Thursdays, 8:30 AM - 10:00 AM, Alden Hall, 109\\


\noindent
\textbf{Laboratory Session}:\\
Duration: 21 Feb 2021 - 20 May 2021\\
Wednesday, 2:50 PM - 4:40 PM, Alden Hall, 109\\

%%%%
\omitit{
\subsection*{\textbf{Schedule}}
The course schedule will be made available on the course website (\url{http://www.cs.allegheny.edu/sites/obonhamcarter/}).
} % end of omitit{}


\subsection*{\textbf{Calendar}}
The calendar link is provided below to allow you to add the course and lab meeting times into your own Google calendar. Note, the whole link fits onto one line.\\
{\footnotesize
\url{https://calendar.google.com/calendar/u/0?cid=Y185NjFwYzMycHU2aGFpOW5tMW9vb3Fnc3Q1c0Bncm91cC5jYWxlbmRhci5nb29nbGUuY29t} }\\
Note: When copying and pasting the above hyperlink for the address, there are no spaces in the link.


\subsection*{\textbf{Discord Channel}}
The below link will expire in 7 days from 21$^{st}$ Feb 2022\\
{\footnotesize
\url{https://discord.gg/8AR35Z3Z} }



\subsection*{\textbf{The {\tt ClassDocs/} Repository}}
All materials given out in class will be accessible using the {\tt classDocs/} repository. Note: The HTTP link works in absence of SSH keys.

\textbf{Main site on GitHub}: 
	\begin{itemize}
		\item \footnotesize \url{https://github.com/Allegheny-ComputerScience-580-S2022/classDocs}
	\end{itemize}

\textbf{HTTPS}: 
	\begin{itemize}
		\item {\tt \footnotesize git clone https://github.com/Allegheny-ComputerScience-580-S2022/classDocs.git}
	\end{itemize}

\textbf{SSH}: 
	\begin{itemize}
		\item {\tt \footnotesize git clone git@github.com:Allegheny-ComputerScience-580-S2022/classDocs.git}
	\end{itemize}



\subsection*{\textbf{Academic Bulletin Description}}

\begin{quote}

A team-based investigation of select topics in computer science, preparing students for the proposal and completion of a senior project. Working in teams to complete hands-on activities, students learn how to read research papers, state and motivate research questions, design and conduct experiments, and collect and organize evidence for evaluating scientific hypotheses. During a weekly laboratory session students use state-of-the-art technology to gain practical skills in scientific and technical writing, the presentation of computational and mathematical concepts, and the visualization of experimental data. Students are invited to use their own departmentally approved laptop in this course; a limited number of laptops are available for use during class and lab sessions.\\

Prerequisite: CMPSC 101 and at least one of the Fundamentals courses.

\end{quote}




\subsection*{\textbf{Suggested Reading}}

The below reading list is strongly recommended to improve students build technical writing skills and to gain a firm understanding in how to conduct responsible research in computer science.


\begin{itemize}

\item Deetjen, Thomas A.. {\em Published: A Guide to Literature Review, Outlining, Experimenting, Visualization, Writing, Editing, and Peer Review for Your First Scientific Journal Article.} Poland: Productive Academic Press, 2020.  ISBN: 9781734493108

\item Dupré, L. (2000). {\em  BUGS in Writing: A Guide to Debugging Your Prose.} United States: Addison-Wesley. 


\item Evans, D., Zobel, J., Gruba, P. (2014). {\em How to Write a Better Thesis.} Germany: Springer International Publishing.

\item Gruba, P., Zobel, J. (2017). {\em How To Write Your First Thesis.} Germany: Springer International Publishing. ISBN: 978-1-4471-6638-2

%\item  \emph{On Being a Scientist: A Guide to Responsible Conduct in Research (Third Edition)}. Committee on Science, Engineering, and Public Policy, National Academy of Sciences, National Academy of Engineering, and Institute of Medicine. ISBN: 0309119715, 82 pages, 2009. References to the textbook are abbreviated as ``OBAS''.

\item Along with reading the required books, you will be asked to study many additional articles from a wide variety of conference proceedings, journals, and the popular press.
\end{itemize}






\subsection*{\textbf{Grading}}

The grade that a student receives in this class will be based on the following categories. All percentages are approximate and, if the need to do so presents itself, it is possible for the assigned percentages to change during the academic semester. 
\color{red}

\begin{center}
  \begin{tabular}{l|l}
\hline
	Class Participation & 10\%\\
	Assignments and lab assignments & 50\% \\
	Research project and associated materials & 40\% \\
%prototypes, research (web) notebooks, reviews of articles, open source software
% demonstrate knowledge of reviewing papers, making a prototype, simple steps for research projects. 
\hline
  \end{tabular}
\end{center}


\color{black}
\noindent


\noindent
\subsection*{\textbf{Definitions of Grading Categories}}
\vspace*{-.05in}

\begin{itemize}

  \itemsep 0em

  \item {\em Class Participation:} All students are required to actively participate in class and lab sessions. Your participation may take the form of contributing to class discussions, completing activities, giving presentations and similar types of class events.  In many cases,  your participation grade will be determined from your timely submissions for your work.  You will likely receive a check-mark grade from the submission of some of your submissions.

%In addition, students should regularly participate in the discussions on the relevant channels in the Slack site for our course. Your participation on Slack may involve giving a quick status update, inviting your instructor to examine a draft of your proposal or compile, or, within the bounds of the Honor Code, answering a question from another junior working on a research proposal. 

  \item {\em  Assignments and lab assignments:} Several assignments will be given to help students gain experience with specific technical and research resources.

%  \item {\em Proposal Deliverables and a Final Proposal:} Each student must complete several proposal deliverables, including writing drafts of certain sections of the proposal, revising their drafts, and peer editing drafts of others. At the end of the semester students must present their proposal and submit a final proposal document that has gone through multiple revisions. 

\item {\em Research project and associated materials:} The student will preparing a proposal in original research in this class.  There will be several assignment which teach focused aspects of this preparing for research projects.

\omitit{
\item {\em Research Notebook and Meeting Records}: All students must keep a research notebook throughout the semester. Your notebook will contain your observations about all of the reading assignments and details about your own research interests. Each of the dated and signed notebook entries should include paragraphs, diagrams, lists of important points and relevant questions, links to Web sites, description of software installation procedures, and other information about research in computer science. Your research notebook will be collected and graded at the end of each module. For every course module, students are asked to attend a fifteen minute meeting with the instructor who coordinates that module. After your meeting with this individual, you must record the date, time, meeting subject, and receive the signature of the professor. Your meeting record will be collected at the end of the semester and a lack of signatures will lead to a reduction of your score for this part of your grade.

\item {\em Writing and Practical Skill Assignments:} For each of the assigned research articles, a student will be responsible for completing a paper review containing short paragraphs that summarize and evaluate the paper and then propose interesting questions, insights, and areas for future work suggested by the reading. For each of the reading assignments from WFCS, OBAS, and BIW, you must write a precis, or a ``concise summary of essential points, statements, or facts'' about the assignment (Merriam-Webster Online Dictionary). During each module of the course, students also must complete a wide variety of practical skill assignments (e.g., writing technical papers in LaTeX, using the ACM Digital Library and BibTeX, creating technical diagrams with Graphviz and PGF/Ti\emph{k}Z, formatting algorithms and equations in LaTeX, and installing open source software). Evidence that a student has mastered the practical skills taught during the module must be evident in the module proposal that the student submits.

\item {\em Research Presentations:} During each module of the course, a team of students will give a fifteen to twenty minute presentation explaining the assigned article(s) and suggesting areas for future research. On the last day of the third and fourth weeks of a module, every student will give a \emph{lightning talk} -- a short three to five minute presentation, leveraging at least two slides, that effectively describes the topic. With a topic that is (roughly) distinct from the theme of the module, the first lightning talk suggest an idea connected to the current module's theme. The second lightning talk must must explore a new idea for your senior thesis.


\item {\em Module Proposals and Presentations:} Using \LaTeX, BibTeX, Vim, and other relevant technical writing tools, a student is responsible for creating a five page proposal and a ten minute presentation that both connect to the theme of each module. The proposal should contain an interesting and informative title, a one paragraph abstract, and several sections of text that describe your proposed research. The presentation should have the same title as your proposal and contain enough slides for a short, yet intuitive and compelling, introduction to your idea.  Students are encouraged to meet with the course instructor and \mbox{the module} professor about their proposed research. The proposal and the presentation slides must contain evidence that the student can use all of the practical skills that were taught during the module.  %The proposal and the presentation slides are due on the fourth Thursday of \mbox{the module}.
}

\end{itemize}











%\vspace{-.1in}
%\subsection*{\textbf{Assignment Completion}}

%All assignments will have a stated due date. To accommodate for unforeseen life events, each student will be given an option of dropping one assignment grade at the end of the semester. The dropped grade cannot include the final proposal assignment. Otherwise, unless severe extenuating circumstances have been presented to the instructor, no assignments will be accepted after the deadline.

%All assignments will have a stated due date.  Late assignments will be accepted for up to one week past the assigned due date with a 15\% penalty. All late work must be submitted at the beginning of the session that is scheduled one week after the due date. Unless special arrangements are made with the course instructor, no assignments will be accepted after the late deadline.


%%%%
\omitit{
\vspace{-.10in}
\subsection*{\textbf{Laboratory Attendance}}

In order to acquire the proper skills in technical writing, critical reading, and the presentation and evaluation of technical material, it is essential for students to have hands-on experience in a laboratory. Therefore, it is mandatory for all students to attend the laboratory sessions. If you will not be able to attend a laboratory, then please see the one of the course instructor at least one week in advance in order to explain your situation. Students who miss more than two unexcused laboratories will have their final grade in the course reduced by one letter grade.  Students who miss more than four unexcused laboratories will automatically fail the course.
}%%%%







\vspace{-.10in}
\subsection*{\textbf{Extensions}}
Unless special arrangements are made with the course instructor, no assignments will be accepted after the late deadline. If you are requesting extensions for an assignment, then you are to email me with your request and also provide a \emph{valid reason} for your extension. This request must come before the due date of the lab and not on the due date. Requests will not be granted where the reason appears to be insignificant. Extensions are 24 hours of extra time (after the original due date) and are given out at my discretion. The decision to provide you with an extension (or not) will be weighed in light of fairness to your peers who are still able to complete their labs, regardless of their own busy schedules. 

\vspace{-.10in}
\subsection*{\textbf{A Note on extenuating circumstances}}

If you should find yourself in difficult circumstances that significantly interfere with your ability to prepare for this class and to complete assignments, please inform me immediately so that we can work something out together! Do not wait until the last day of class to ask for exceptions to what is stated in this syllabus. In such a situation, you may also find it helpful to contact one of the available resources on campus: 
\begin{itemize}
	\item The Maytum Learning Commons, Library/Academic Commons,\\ \url{http://sites.allegheny.edu/learningcommons/tutoring/},\\ 814-332-2898
%You may request an individual tutor through Learning Commons by visiting,

	\item Counseling \& Personal Development Center,\\ \url{https://sites.allegheny.edu/counseling/},\\ 814-332-2105
	\item Winslow Health Center,\\ \url{https://sites.allegheny.edu/healthcenter/},\\ 814-332-4355
\end{itemize}

\subsection*{\textbf{Communication}}
Various digital channels will be used in this course for communication, including email,
Discord, and the GitHub issue tracker. It is strongly advised for the student to install the Discord app on their computer and smart-phone to be sure to receive all communications from the instructor, as well as, the other members of the class.

Additionally, the course website will be used to store the syllabus, course schedule and information about the {\tt classDocs/} repository using the GitHub.  Your grades will be communicated to you by a Gradebook GitHub repository. 




\vspace*{-.10in}
\subsection*{\textbf{Special Needs and Disability Services}}

The Americans with Disabilities Act (ADA) is a federal anti-discrimination statute that provides comprehensive civil rights protection for persons with disabilities.  Among other things, this legislation requires all students with disabilities be guaranteed a learning environment that provides for reasonable accommodation of their disabilities. Students with disabilities who believe they may need accommodations in this class are encouraged to contact Disability Services at 332-2898. Disability Services is part of the Learning Commons and is located in Pelletier Library. Please do this as soon as possible to ensure that approved accommodations are implemented in a timely fashion.

\vspace{-.10in}
\subsection*{\textbf{Honor Code}}

The Academic Honor Program that governs the entire academic program at Allegheny College is described in the Allegheny Course Catalogue.  The Honor Program applies to all work that is submitted for academic credit or to meet non-credit requirements for graduation at Allegheny College.  This includes all work assigned for this class (e.g., examinations, laboratory assignments, and the final project).  All students who have enrolled in the College will work under the Honor Program.  Each student who has matriculated at the College has acknowledged the following pledge:

\vspace*{-.1in}
\begin{quote}
\emph{I hereby recognize and pledge to fulfill my responsibilities, as defined in the Honor Code, and to maintain the integrity of both myself and the College community as a whole.}
\end{quote}
\vspace*{-.15in}

\noindent It is recognized that an important part of the learning process in any course, and particularly one in computer science, derives from thoughtful discussions with teachers and fellow students.  Such dialogue is encouraged. However, it is necessary to distinguish carefully between the student who discusses the principles underlying a problem with others and the student who produces assignments that are identical to, or merely variations on, someone else's work.  While it is acceptable for students in this class to discuss their programs, technical diagrams, proposals, paper reviews, presentations, and other items with their classmates or other individuals, deliverables that are nearly identical to the work of others will be taken as evidence of violating the \mbox{Honor Code}.




\end{document}
%%%% junk bin %%%%
%%%% junk bin %%%%
%%%% junk bin %%%%
%%%% junk bin %%%%




\omitit{
%\vspace{-.10in}
\subsection*{\textbf{Course Schedule}}

\subsection*{\textbf{Overview}}

This class is divided into four modules: a two-week introduction to research in \mbox{computer science and} three four-week modules focusing on the introduction of both distinct areas of \mbox{computer science and} the practical and conceptual skills needed to conduct research in the field.  This schedule is preliminary and, if the need to do so presents itself, it is possible for it to change during the semester.
}


\begin{figure}[ht!]
\begin{center}
\includegraphics[scale=.9]{graphics/cvirus.jpg}
\end{center}
\caption{Safety first: Face masks and social distancing in effect.}
\label{fig:cvirus}
\end{figure}

The pandemic from the coronavirus (shown in Figure \ref{fig:cvirus}) has changed the usual style of teaching of this course. Please follow the below points carefully. 
\begin{itemize}

\item \textbf{Remote Attendance}:
If you are participating entirely remotely this semester and relying on technology to attend class meetings, occasional technology problems that disrupt your participation will not harm your participation grade, but as with illnesses and family emergencies, chronic absences for this reason will require a more extensive discussion with me and may impact your grade.

\item \textbf{Face Coverings and Physical Distancing}:
For your safety, a mask covering both your mouth and your nose is required for all in-person activities, per College policy; you will not be permitted to enter or stay in a classroom or other learning space without a face covering, and class time missed for this reason may count against your participation grade. Face coverings are also required for in-person office hours and consultations with other campus professionals. Physical distancing must be respected at all times in the classroom. Chairs will be positioned 6 feet apart, and should remain so.


\item \textbf{Illness and In-person Attendance}:
If you feel ill, please stay in your residence and complete the daily health screening, and err on the side of caution when deciding whether or not to come to class. Especially if you feel feverish or have a cough, please avoid contact with others; if you feel like you'd like to ``power through" class rather than miss it and have to make it up, please do so remotely. 

\item \textbf{Keeping Devices Charged}:
You will need to ensure that your laptop, tablet, or other device is sufficiently charged so that you may participate in class(es). Even if you are in-person in the classroom, you may need to use a device, especially as you will be 6 feet from your nearest peer. It won’t be possible for all students to charge their devices at once in the classroom, so please make sure you bring the power cord(s) for your devices to class, pack a power strip if you have multiple devices, and pay attention to the power meter on your device. 

\item \textbf{Video and Microphones}:
%In addition, faculty may want to consider a statement on \emph{netiquette} (use of microphone, video, protocols for how to participate in video conference meetings).
Please turn off your microphone when not speaking during any meeting where you are using your computer. The microphone may allow for background sound to contribute to noise during the meeting. It is strongly encouraged that you use your video to show yourself during meeting. Enabling your video will allow the instructor to see hands to indicate questions. Showing video also helps to stimulate group discussions.
\end{itemize}




\subsection*{\textbf{Welcome to Computer Science Research!}}

Computer hardware and software are everywhere! Conducting research in computer science is a challenging and rewarding activity that leads to the production of hardware, software, and scientific insights that have the potential to positively influence the lives of many people.  As you learn more about research methods in computer science you will also enhance your ability to effectively write and speak about a wide range of topics in computer science. I ask that you bring your best effort and highest enthusiasm as you pursue research in computer science this semester. 



\end{document}



%%%% junk bin %%%%
%%%% junk bin %%%%
%%%% junk bin %%%%
%%%% junk bin %%%%

%\section*{\textbf{Class Policies}}


%\vspace{-.1in}
%\subsection*{\textbf{Class Attendance}}

%It is important to attend all of the class and laboratory sessions. However, due to the hybrid and sometimes asynchronous mode of teaching, attendance is not expected. Instead, class exercises will be used to evaluate participation.

\color{red}
\subsection*{\textbf{Hybrid Class Sessions}} Due to the COVID-19 pandemic (and the size of our classroom), the class will be divided into two groups to allow ample space between members of the class. Groups alternate between attending a Tuesday or Thursday online or in-person. For instance, the Tuesday in-person group will attend Tuesday classes in-person, while the Thursday group will attend online using Zoom. On Thursdays, the groups will switch and the Tuesday group will attend online, while the Thursday group is in person. It is mandatory for all students to attend his or her scheduled class, as appropriate. \textbf{Switching groups without instructor permission will not be allowed due to regulations for classroom spacing.}


If you will not be able to attend your session, then please email the course instructor at least one week in advance to describe your situation.  Students who miss more than five unexcused classes, laboratory sessions, or group project meetings will have their final grade in the course reduced by one letter grade. Students who miss more than ten of the aforementioned events will automatically fail the course.
\color{black}

\

